Lorsqu’un utilisateur prend en main un outil, il doit savoir comment l’utiliser. Pour cette raison, chaque outil est souvent accompagné d’un manuel d’utilisation. Celui-ci contient généralement une liste des fonctionnalités de l’outil, présentée de manière à ce que l’utilisateur sache comment en profiter.

Ce guide d’utilisation pour ChromaCase permet à son utilisateur d’accéder et de prendre en main l’application, en lui illustrant chacune de ses fonctionnalités.
En lisant ce manuel, l’utilisateur aura toutes les ressources nécessaires pour se connecter à l’application, chercher un artiste ou une chanson, lancer une partie, et customiser son interface personnelle. De plus, ce document présente comment l’application va récupérer les notes jouées. A la fin de cette documentation, le lecteur pourra retrouver une liste de questions fréquemment posées, ainsi qu’un lien de contact, afin d’obtenir plus d'informations sur ChromaCase.

Cette documentation utilisateur permettra donc à son lecteur de comprendre le fonctionnement de l’application, comment en tirer le meilleur parti pour avoir la meilleure expérience possible.
