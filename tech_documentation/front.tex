Le code source de l’application front est dans le  dossier \verb|./front|.
L’application Front-end est développée en React-Native, bootstrapé avec Expo. Ce sont les seules technologies stables permettant d'avoir une seule code-base pour un projet à la foix web et mobile. De plus, de nombreuses librairies tierces sont disponibles. 
\\\\
Concernant les communications avec l’API, les requêtes sont encapsulées dans le fichier API.ts. Les requêtes passent par les fonctions utilitaires de react-query (ex: \verb|useQuery|) qui permet de gérer facilement les états de chargement et d’erreurs, dans un paradigme réactif.
\\\\
L’application utilise un store Redux pour persister, rehydrater, et hook-er des paramètres de connexion (ex: access token) et les paramètres utilisateurs (thème, language). Cette librairie est tres populaire et stable. 
\\\\
L’identité visuelle et le style associés sont mis en place grâce à la bibliothèque de style native-base. La configuration du thème est dans Theme.tsx. Le hook dans \verb|./hooks/colorScheme.ts| permet d’accéder à la configuration du thème (clair ou foncé) en fonction des paramètres de l’application et du navigateur.
\\\\
La navigation est gérée par react-navigation. Le fichier \verb|Navigation.tsx| met en place les routes disponibles en fonction de l’état de connexion de l’utilisateur (anonyme ou connecté).
\\\\
Les types des réponses de l’API sont validées au runtime grâce à la bibliothèque YUP. Les validateurs et modèles associées sont dans le dossier \verb|./models|. Cette librairie a ete choisie car elle est facile d'utilisation, d'integration et à maintenir dans le projet.
\\\\
Le support des traductions (Français, Anglais, Espagnol) sont faites avec le package react-i18n. Les valeurs des traductions sont dans le dossier \verb|./i18n|.
\\\\
Concernant la connexion au piano MIDI, l’application utilise l’API MIDI supportée par le navigateur et les mobiles.
\\\\
La gestion des paquets se fait avec Yarn, car il permet une meilleure gestion des sous-dependences que npm.
