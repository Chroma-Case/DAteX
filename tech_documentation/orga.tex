Le code source du projet est disponible sur \href{https://github.com/Chroma-Case/Chromacase}{https://github.com/Chroma-Case/Chromacase}.
La plupart de l’organisation se fait sur ce dépôt.
\\\\
Les issues GitHub permettent de pister l’avancement d’une tâche. Chacune d’elles sont associées à un composant sur projet (Front, server, scorometer) via les GitHub Projects. De plus, chacune des tâches sont attribuées à un.e développeur.euse.
Il n’est pas possible de directement commit des changements sur la branche principale main. Pour chaque tâche, le travail/code associées doit être sur une branche dédiée.
Lorsqu’une tâche est terminée, la personne en charge doit ouvrir une Pull Request avant que le nouveau code soit ajouté sur la branche principale. Il ne sera possible de fusionner les 2 branches si, et seulement si au moins 2 autres contributeurs ont review la PR, et la CI passe (c.f. section CI)
\\\\
Les contributeurs du projet communiquent sur Discord. Chaque composant du projet dispose d’un channel textuel et audio qui lui est propre. Il existe également un channel Général pour la communication générale, et Notes réunions pour reporter le compte-rendu des réunions, au cas où un contributeur ne peut pas y assister.
\\\\
Des réunions sont organisées toutes les 2 semaines.
