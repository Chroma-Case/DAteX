\subsection{Utilisation du dépôt GitHub}
	Le code source du projet est disponible sur \url{https://github.com/Chroma-Case/Chromacase}.
	La plupart de l’organisation se fait sur ce dépôt.
	\\\\
	Les \textbf{issues GitHub} permettent de pister l’avancement d’une tâche. Chacune d’elles sont associées à un composant sur projet (Front, server, scorometer) via les \textbf{GitHub Projects}. De plus, chacune des tâches sont attribuées à un.e développeur.euse.
	Il n’est pas possible de directement commit des changements sur la branche principale main. Pour chaque tâche, le travail/code associé doit être sur une branche dédiée.
	Lorsqu’une tâche est terminée, la personne en charge doit ouvrir une Pull Request avant que le nouveau code soit ajouté sur la branche principale (main). Il ne sera possible de fusionner les 2 branches si, et seulement si au moins deux autres contributeurs ont review la PR, et la CI passe (c.f. section Intégration Continue)

\subsection{Utilisation de Discord}
	Les contributeurs du projet communiquent sur Discord. Chaque composant du projet dispose d’un channel textuel et audio qui lui est propre (cf figure \ref{fig:discord-chans}). Il existe également un channel 'Général' pour la communication générale, et 'Notes réunions' pour reporter le compte rendu des réunions, au cas où un contributeur ne peut pas y assister.

	Le serveur Discord est privé et doit être utilisé exclusivement dans le cadre du projet. Vous recevrez son lien d'accès lors de votre intégration dans l'équipe.

	\begin{figure}[H]
		\begin{itemize}
			\item Channels d'organisation:
			\begin{itemize} 
				\item Général
				\item Notes de réunions
				\item Testing
				\item Documentation
			\end{itemize}
			\item Channels techniques:
			\begin{itemize} 
				\item Front
				\item Scorometer
				\item Back
				\item CI/CD
				\item Design
			\end{itemize}
		\end{itemize}
		\caption{Liste des channels de communication du server Discord}
		\label{fig:discord-chans}
	\end{figure}

\subsection{Mise en commun et conciliation}
	Des réunions sont organisées toutes les deux semaines. Celles-ci sont instanciées sur Discord, sous la forme d'évent, avec une date et une heure. Quand un évent est créé, un message mentionnant tous les membres du server doit être posté pour l'annoncer.
