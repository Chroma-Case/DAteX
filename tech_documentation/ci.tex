Pour chaque changement commit fait sur chaque branche, les vérifications suivantes sont faites pour chaque composants du projet, via un GitHub Action (le fichier associé se trouve dans \verb|./.github/workflows/CI.yml|):

\begin{itemize}
	\item Build du Server
	\item Build du Scorometer
	\item Build du Front
	\item Vérification de la norme (Linter et Prettier) du Server
	\item Vérification de la norme du Front
	\item Vérification de la norme du Scorometer
	\item Exécutions des tests du Server
	\item Exécutions des tests du Tests
\end{itemize}


Ces vérifications permettent d’assurer que le nouveau code soit valide, et ne ‘casse’ pas la production.
La validation de chacune de ces étapes pour chaque composant du projet est nécessaire avant de pouvoir fusionner une branche sur main, et ainsi créer une release du projet.

Le choix des GitHub Actions pour gérer la CI se justifie par leur facilité d'utilisation et leur integration facile avec les issues, pour traquer leur avancement.