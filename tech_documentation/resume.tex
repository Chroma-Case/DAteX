Tout projet informatique doit être délivré avec 2 types de documentation: une documentation destinée à l’utilisateur du produit final, et une documentation destinée aux développeurs du projet. Cette dernière existe afin de permettre de aux potentiels nouveaux développeurs arrivant sur le projet de comprendre la structure, le fonctionnement et les normes que suit le projet.

Dans ce document, le lecteur pourra retrouver une description détaillée des de la structure du projet ChromaCase, de ses composants, c’est-à-dire de sa base de donnée, du serveur, de l’application front-end et du scorometer, ainsi que la manière dont ses composants communiquent entre eux. De plus, en lisant ce document, le développeur retrouvera comment prendre contact avec les différentes équipes du projet, et comment celles-ci coopèrent. Enfin, ce document explique le workflow de développement, permettant au développeur de comprendre quelles sont les étapes à suivre pour que ses contributions soient considérées et acceptées.

Grâce à ce document, le lecteur aura tous les éléments nécessaires pour comprendre le fonctionnement et collaborer avec l’équipe ChromaCase.
