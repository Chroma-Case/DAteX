Tout projet informatique doit être délivré avec deux types de documentation: une documentation destinée à l’utilisateur du produit final et une documentation destinée aux développeurs du projet. Cette dernière vise à permettre aux nouveaux développeurs arrivant sur le projet de comprendre la structure et le design du projet, le fonctionnement des blocs qui le compose et les interactions que ces blocs ont entre eux.

Le projet ChromaCase repose sur une structure en micro-service. Chaque micro-service, le choix des technologies utilisées, leur fonctionnement et architecture sont décrits dans ce document.

Le développement du projet suit des règles strictes inférées par une CI/CD et des vérifications régulières de coding-style et norme (Lint, Prettier).

À la fin de la lecture de ce document, le lecteur sera capable de comprendre le fonctionnement interne du projet et y contribuer.