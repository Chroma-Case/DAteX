Le code source du server est dans le dossier \texttt{./back}. Le serveur est développé en TypeScript, avec le framework NestJS. Ce dernier a ete choisit car il permet de gerer facilement et rapidement des CRUD.
\\\\
L’ORM utilisé est Prisma. La définition des schéma se trouve dans \texttt{./prisma/schema.prisma}. (cf. Section “Base de données”). L'ORM a été choisit car il permet de maintenir une type-safety tout le long de l'utilisation des données extraites de la BDD
\\\\
L’application est basée sur l'injection de dépendances. Les modules sont les suivants:

\begin{itemize}
	\item \texttt{UsersModule}
	\item \texttt{PrismaModule}
	\item \texttt{AuthModule}
	\item \texttt{SongModule}
	\item \texttt{LessonModule}
	\item \texttt{GenreModule}
	\item \texttt{ArtistModule}
	\item \texttt{AlbumModule}
	\item \texttt{SearchModule}
	\item \texttt{SettingsModule}
	\item \texttt{HistoryModule}
\end{itemize}
\  \\ % Forcing a line jump after list
Excepté pour le \texttt{PrismaModule} et le \texttt{AuthModule}, chaque module expose un controller et un service de type CRUD.
Le \texttt{PrismaModule} est chargé de la connexion avec la base de données. Le \texttt{AuthModule} est responsable de l'authentification de l’utilisateur qui fait la requête en utilisant un JWT fourni avec cette dernière.
\\\\
La documentation de l’API du serveur se présente sous la forme d’un Swagger accessible depuis \url{https://nightly.chroma.octohub.app/api/api}
\\\\
La gestion des paquets se fait avec NPM, car il est plus facilement utilisable que yarn pour NestJS.
\\\\
Les tests du serveur sont des tests End-To-End (E2E) faits en Robot, qui testent les méthodes de CRUD (Creation, Read, Update, Delete) des controllers. 
