Le code source du server est dans le dossier \texttt{./back}. Le serveur est développé en TypeScript, avec le framework NestJS. Ce dernier a été choisit, car il permet de gérer facilement et rapidement des CRUD.

\subsection*{Relation avec la base de données}
L’ORM utilisé est Prisma. La définition des schéma se trouve dans \texttt{./prisma/schema.prisma}. (cf. Section “Base de données”). L'ORM a été choisit car il permet de maintenir une type-safety tout le long de l'utilisation des données extraites de la BDD

\subsection*{Modules}
L’application est basée sur l'injection de dépendances. Les modules sont les suivants:

\begin{itemize}
	\item \texttt{UsersModule}
	\item \texttt{PrismaModule}
	\item \texttt{AuthModule}
	\item \texttt{SongModule}
	\item \texttt{LessonModule}
	\item \texttt{GenreModule}
	\item \texttt{ArtistModule}
	\item \texttt{AlbumModule}
	\item \texttt{SearchModule}
	\item \texttt{SettingsModule}
	\item \texttt{HistoryModule}
\end{itemize}
\  \\ % Forcing a line jump after list
Excepté pour le \texttt{PrismaModule} et le \texttt{AuthModule}, chaque module expose un controlleur et un service de type CRUD.
Le \texttt{PrismaModule} est chargé de la connexion avec la base de données. Le \texttt{AuthModule} est responsable de l'authentification de l’utilisateur qui fait la requête en utilisant un JWT fourni avec cette dernière.

\subsection*{API}

L'API du back-end propose les routes suivantes

\begin{itemize}
	\item \verb|/users|: Récupérer les utilisateurs inscrits
	\item \verb|/users/:id|: Récupérer un utilisateur
	\item \verb|/users/:id/picture|: Récupérer l'avatar d'un utilisateur
	\item \verb|/auth/login/google|: Se connecter en utilisant un token Google
	\item \verb|/auth/register|: Créer un compte
	\item \verb|/auth/login|: Se connecter
	\item \verb|/auth/guest|: Se connecter en tant qu'invité
	\item \verb|/auth/me|: Récupérer les informations de l'utilisateur authentifié
	\item \verb|/auth/me/picture|: Récupérer l'avatar de l'utilisateur authentifié
	\item \verb|/auth/me/settings|: Récupérer les préférences de l'utilisateur authentifié
	\item \verb|/song|: Récupérer toutes les chansons du catalogue
	\item \verb|/song/{id}|: Récupérer une chanson du catalogue
	\item \verb|/song/{id}/midi|: Récupérer le fichier midi d'une chanson
	\item \verb|/song/{id}/musicXml|: Récupérer le fichier MusicXml d'une chanson
	\item \verb|/song/{id}/illustration|: Récupérer l'illustration d'une chanson
	\item \verb|/song/{id}/history|: Récupérer l'historique de partie sur une chanson
	\item \verb|/history|(GET): Récupérer l'historique de partie de l'utilisateur authentifié
	\item \verb|/history|(POST): Ajouter une entrée l'historique de partie de l'utilisateur authentifié
	\item \verb|/genre|: Récupérer tous les genres
	\item \verb|/genre/{id}|: Récupérer un genre
	\item \verb|/genre/{id}/illustration|: Récupérer l'illustration d'un genre
	\item \verb|/artist|: Récupérer tous les artistes du catalogue
	\item \verb|/artist/{id}|: Récupérer un artiste du catalogue
	\item \verb|/artist/{id}/illustration|: Récupérer l'illustration d'un artiste
	\item \verb|/search/songs/{query}|: Rechercher une chanson, basé sur son nom
	\item \verb|/search/genres/{query}|: Rechercher un genre, basé sur son nom
	\item \verb|/search/artists/{query}|: Rechercher un artiste, basé sur son nom
\end{itemize}

La documentation de l’API du serveur se présente sous la forme d’un Swagger accessible depuis \url{https://chroma.octohub.app/api/api}

\subsection*{Packet Management}
La gestion des paquets se fait avec NPM, car il est plus facilement utilisable que yarn pour NestJS.

\subsection*{Tests}
Les tests du serveur sont des tests End-To-End (E2E) faits en Robot, qui testent les méthodes de CRUD (Creation, Read, Update, Delete) des controlleurs. 
