Le nombre de contributeurs.euses sur le projet étant important, il est important que chacun.e suivent la même norme de code afin d’avoir une code base homogène.
Pour cette raison, ont été mis en place un Linter (eslint) et un Prettier (prettier) qui permettent respectivement d'identifier les erreurs de norme (définies au préalable) et d’appliquer la correcte indentation et sauts de ligne.
Ces deux outils ont été choisis car elles sont les plus populaires dans leurs domaines, et sont faciles à intégrer et utiliser.
\\\\
Le fichier de configuration eslint est \verb|./front/.eslintrc.json|. La documentation des règles utilisées est disponible sur le site d’eslint.
Entre autre, les règles suivantes doivent être respectées:

\begin{itemize}
	\item Pas de \verb|any|
	\item Pas de variables non utilisée
	\item Pas d'utilisation de l'operateur \verb|!|
\end{itemize}

Le fichier de configuration prettier est \verb|./front/.prettierrc.json|. La documentation des règles utilisées est disponible sur le site de prettier.
Les règles appliquées sont:
\begin{itemize}
	\item Indentation avec des tabulations (taille 4)
	\item Chaque expression doit être ponctuée d'un \verb|;|
	\item Des espaces avant et après les accolades
	\item Utilisation de simples quotes
	\item Taille Maximale des lignes: 100 colonnes
\end{itemize}